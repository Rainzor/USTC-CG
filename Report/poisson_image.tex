\documentclass[14pt]{scrartcl} % Font size

%%%%%%%%%%%%%%%%%%%%%%%%%%%%%%%%%%%%%%%%%
% Wenneker Assignment
% Structure Specification File
% Version 2.0 (12/1/2019)
%
% This template originates from:
% http://www.LaTeXTemplates.com
%
% Authors:
% Vel (vel@LaTeXTemplates.com)
% Frits Wenneker
%
% License:
% CC BY-NC-SA 3.0 (http://creativecommons.org/licenses/by-nc-sa/3.0/)
% 
%%%%%%%%%%%%%%%%%%%%%%%%%%%%%%%%%%%%%%%%%

%----------------------------------------------------------------------------------------
%	PACKAGES AND OTHER DOCUMENT CONFIGURATIONS
%----------------------------------------------------------------------------------------

\usepackage{amsmath, amsfonts, amsthm} % Math packages

\usepackage[utf8]{inputenc}

\usepackage[UTF8]{ctex}

\usepackage{listings} % Code listings, with syntax highlighting

\usepackage{subfig}

\usepackage[english]{babel} % English language hyphenation

\usepackage{amssymb}

\usepackage{graphicx} % Required for inserting images
\graphicspath{{Figures/}{./}} % Specifies where to look for included images (trailing slash required)

\usepackage{booktabs} % Required for better horizontal rules in tables

\numberwithin{equation}{section} % Number equations within sections (i.e. 1.1, 1.2, 2.1, 2.2 instead of 1, 2, 3, 4)
\numberwithin{figure}{section} % Number figures within sections (i.e. 1.1, 1.2, 2.1, 2.2 instead of 1, 2, 3, 4)
\numberwithin{table}{section} % Number tables within sections (i.e. 1.1, 1.2, 2.1, 2.2 instead of 1, 2, 3, 4)

\setlength\parindent{0pt} % Removes all indentation from paragraphs

\usepackage{enumitem} % Required for list customisation
\setlist{noitemsep} % No spacing between list items

%----------------------------------------------------------------------------------------
%	DOCUMENT MARGINS
%----------------------------------------------------------------------------------------

\usepackage{geometry} % Required for adjusting page dimensions and margins

\geometry{
	paper=a4paper, % Paper size, change to letterpaper for US letter size
	top=2.5cm, % Top margin
	bottom=3cm, % Bottom margin
	left=3cm, % Left margin
	right=3cm, % Right margin
	headheight=0.75cm, % Header height
	footskip=1.5cm, % Space from the bottom margin to the baseline of the footer
	headsep=0.75cm, % Space from the top margin to the baseline of the header
	%showframe, % Uncomment to show how the type block is set on the page
}

%----------------------------------------------------------------------------------------
%	FONTS
%----------------------------------------------------------------------------------------

\usepackage[utf8]{inputenc} % Required for inputting international characters
\usepackage[T1]{fontenc} % Use 8-bit encoding

\usepackage{fourier} % Use the Adobe Utopia font for the document

%----------------------------------------------------------------------------------------
%	SECTION TITLES
%----------------------------------------------------------------------------------------

\usepackage{sectsty} % Allows customising section commands

\sectionfont{\vspace{6pt}\centering\normalfont\scshape} % \section{} styling
\subsectionfont{\normalfont\bfseries} % \subsection{} styling
\subsubsectionfont{\normalfont\itshape} % \subsubsection{} styling
\paragraphfont{\normalfont\scshape} % \paragraph{} styling

%----------------------------------------------------------------------------------------
%	HEADERS AND FOOTERS
%----------------------------------------------------------------------------------------

\usepackage{scrlayer-scrpage} % Required for customising headers and footers

\ohead*{} % Right header
\ihead*{} % Left header
\chead*{} % Centre header

\ofoot*{} % Right footer
\ifoot*{} % Left footer
\cfoot*{\pagemark} % Centre footer % Include the file specifying the document structure and custom commands

%----------------------------------------------------------------------------------------
%	TITLE SECTION
%----------------------------------------------------------------------------------------

\title{	
	\normalfont\normalsize
	%\vspace{25pt} % Whitespace
	\rule{\linewidth}{0.5pt}\\ % Thin top horizontal rule
	\vspace{20pt} % Whitespace
	{\huge 实验三	Poisson Image Editing}\\ % The assignment title
	\vspace{12pt} % Whitespace
	\rule{\linewidth}{2pt}\\ % Thick bottom horizontal rule
	\vspace{12pt} % Whitespace
}

\author{\LARGE ID: 58		陈文博} % Your name

\date{\normalsize\today} % Today's date (\today) or a custom date

\begin{document}
	
	\maketitle % Print the title
	
	%----------------------------------------------------------------------------------------
	%	FIGURE EXAMPLE
	%----------------------------------------------------------------------------------------
	
	\section{实验要求}
	
	%\begin{figure}[h] % [h] forces the figure to be output where it is defined in the code (it suppresses floating)
	%	\centering
	%	\includegraphics[width=0.5\columnwidth]{swallow.jpg} % Example image
	%	\caption{European swallow.}
	%\end{figure}
	
	\begin{itemize}
		\item[*] 实现 Poisson Image Editing 算法
		\item[*] 实现多边形光栅化的(扫描线转换算法)
		\item[*] 学习使用Eigen库求解大型稀疏方程组
		\item[*] 学习使用 OpenCV
		\item[*] 实时拖动区域显示结果(Optional)
		\begin{itemize}
			\item 矩阵预分解
		\end{itemize}
	\end{itemize}
	
	%------------------------------------------------
	
	\section{开发环境}
	
	\textbf{IDE}:Microsoft Visual Studio 2019 community\\
	\textbf{CMake}:3.16.3\\
	\textbf{Qt}:5.14.1\\
	\textbf{Eigen}:3.3.7\\
	\textbf{OpenCV}:4.2.0
	
	
	%----------------------------------------------------------------------------------------
	%	TEXT EXAMPLE
	%----------------------------------------------------------------------------------------
	\pagebreak
	\section{算法原理}
	
	\subsection{问题描述}
	\begin{figure}[h] % [h] forces the figure to be output where it is defined in the code (it suppresses floating)
		\begin{minipage}[t]{0.5\linewidth}
			\centering
	%		\includegraphics[width=0.8\linewidth]{girl.jpg}
			\caption{girl}
		\end{minipage}%
		\begin{minipage}[t]{0.5\linewidth}
			\centering
	%		\includegraphics[width=0.8 \linewidth]{sea.jpg}
			\caption{sea}
		\end{minipage}
	\end{figure}
	如上两幅图,现我们需要将Figure 3.1中的女孩搬到Figure 3.2的海水中,为使得复制粘贴更加逼真自然,我们需要设计算法来满足我们两幅图像融合的需要
	
	\subsection{Poisson Image Editing算法\cite{perez2003poisson}}
	
	Poisson Image Editing算法的基本思想是在尽可能保持原图像内部梯度的前提下,让粘贴后图像的边界值与新的背景图相同,以实现无缝粘贴的效果。从数学上讲,对于原图像$f(x,y)$,新背景$f^*(x,y)$和嵌入新背景后的新图像$v(x,y)$,等价于解最优化问题:
	\begin{equation}
		\min\limits_f \iint _\Omega |\nabla f-\nabla \boldsymbol v |^2 \ \ \mathrm{with}\ f|_{\partial \Omega}=f^*|_{\partial \Omega}
	\end{equation}
	利用变分法可转化为具有Dirichlet边界条件的Poisson方程:
	\begin{equation}
		\Delta f= \Delta \boldsymbol v\ \mathrm{over}\ \Omega \ \ \mathrm{with}\ f|_{\partial \Omega}=f^*|_{\partial \Omega}
	\end{equation}
	
	
	以Figure 3.1和Figure 3.2为例,将Figure 3.1中需要复制的区域设为$S$,定义$N_p$为$S$中的每一个像素$p$四个方向连接邻域,令$<p,q>$为满足$q\in N_p$的像素对。边界$\Omega$定义为$\partial \Omega =\{p\in S\setminus \Omega: N_p \cap \Omega \neq \emptyset \}$,设$f_p$为$p$处的像素值$f$,目标即求解像素值集$f|_\Omega =\{f_p,p\in \Omega\}$
	
	利用Poisson Image Editing算法的基本原理,上述问题转化为求解最优化问题:
	\begin{equation}
		\min\limits_{f|_\Omega}\sum\limits_{<p,q>\cap \Omega\neq \emptyset}(f_p-f_q-v_{pq})^2,\mathrm{with}\ f_p=f_p^*,\mathrm{for}\ \mathrm{all}p\in \partial\Omega
	\end{equation}
	
	化为求解线性方程组:
	\begin{equation}
		\mathrm{for}\ \mathrm{all}\ p\in \Omega,\ |N_p|f_p-\sum\limits_{q\in N_p\cap \Omega} f_q=\sum\limits_{q\in N_p\cap \partial \Omega}f_p^*+\sum\limits_{q\in N_p}v_{pq}
	\end{equation}
	对于梯度场$\boldsymbol{v}(\boldsymbol{x})$的选择,文献\cite{perez2003poisson}给出两种方法,一种是完全使用前景图像的内部梯度,即:
	\begin{equation}
		\mathrm{for}\ \mathrm{all}\ <p,q>,v_{pq}=g_p-g_q
	\end{equation}
	另一种是使用混合梯度:
	\begin{equation}
		\mathrm{for}\ \mathrm{all}\ \boldsymbol{x}\in \Omega,\ \boldsymbol{v}(\boldsymbol{x})=\begin{cases}
			\nabla f^*(\boldsymbol{x})&\mathrm{if}\ |\nabla f^*(\boldsymbol{x})>|\nabla g(\boldsymbol{x})|,\\
			\nabla g(\boldsymbol{x})&\mathrm{otherwise}
		\end{cases} 
	\end{equation}
	
	\subsection{扫描线算法}
	为实现多边形和自由绘制闭合图形区域的Poisson Image Editing算法,需通过扫描线算法获取多边形内部掩膜。这里从网上资料了解到一种有序边表法,其基本思想是定义边表ET和活动边表AET,ET记录当前扫描线与边的交点坐标、从当前扫描线到下一条扫描线间x的增量、该边所交的最高扫描线,AET记录只与当前扫描线相交的边的链表,通过迭代得到当前扫描线与待求多边形各边的交点,再利用奇偶检测法判断该点是否在多边形内部进行填充。
	
	
	\pagebreak
	\section{程序架构}
	\subsection{文件结构}
	\begin{figure}[h] % [h] forces the figure to be output where it is defined in the code (it suppresses floating)
		\centering
	%	\includegraphics[width=0.5\columnwidth]{1.png} % Example image
		\caption{文件结构}
	\end{figure}
	
	
	%------------------------------------------------
	\pagebreak
	\subsection{面向对象设计}
	
	\begin{figure}[h] % [h] forces the figure to be output where it is defined in the code (it suppresses floating)
		\centering
	%	\includegraphics[width=1\columnwidth]{class.png} % Example image
		\caption{类图}
	\end{figure}
	
	直接使用实验一MiniDraw中的Shape类进行修改实现Rectangle、Freedraw和Polygon形状的绘制,使用Poisson类实现Poisson Image Editing,ScanLine类实现多边形内部填充算法
	
	%------------------------------------------------
	\pagebreak
	\section{设计难点与解决}
	
	\subsection{OpenCV框架的移植}
	
	使用OpenCV进行图像处理会比直接使用QImage进行像素操作方便很多,在移植过程中要注意各个涉及QImage的环节都要更换为Mat类的等价表示,在显示图像的最后一步将Mat类型转为QImage类型实现Qt上的显示
	%------------------------------------------------
	
	\subsection{图像处理的实时显示}
	
	由于图像在新背景中拖动的时候只改变边界值,即矩阵表示的线性方程组$\boldsymbol A \boldsymbol x=\boldsymbol b$中的$\boldsymbol b$,利用该特性可以采用矩阵预分解减小计算量,经比较采用Eigen的SimplicialLLT求解器,在release模式下可达到实时显示的效果。 
	
	\subsection{关于扫描线算法}
	
	除了利用有序边表的方法实现多边形内部的填充,也可以利用OpenCV的fillPoly同样能够进行多边形内部的填充,经测试可以实现同样的效果。
	
	
	
	\pagebreak
	
	\section{实验效果}
	
	\subsection{标准图像测试}
	
	原图像:
	
	\begin{figure}[h] % [h] forces the figure to be output where it is defined in the code (it suppresses floating)
		\begin{minipage}[t]{0.5\linewidth}
			\centering
	%		\includegraphics[width=0.9\linewidth]{bear.jpg}
			\caption{bear}
		\end{minipage}%
		\begin{minipage}[t]{0.5\linewidth}
			\centering
	%		\includegraphics[width=0.9 \linewidth]{girl.jpg}
			\caption{girl}
		\end{minipage}
	\end{figure}
	
	新背景图像:
	
	\begin{figure}[h] % [h] forces the figure to be output where it is defined in the code (it suppresses floating)
		\centering
	%	\includegraphics[width=0.8\columnwidth]{sea.jpg} % Example image
		\caption{sea}
	\end{figure}
	
	\pagebreak
	\subsubsection{Rectangle}
	
	\begin{figure}[h] % [h] forces the figure to be output where it is defined in the code (it suppresses floating)
		\begin{minipage}[t]{0.5\linewidth}
			\centering
	%		\includegraphics[width=0.9\linewidth]{bearrect.png}
			\caption{bear}
		\end{minipage}%
		\begin{minipage}[t]{0.5\linewidth}
			\centering
	%		\includegraphics[width=0.9 \linewidth]{girlrect.png}
			\caption{girl}
		\end{minipage}
	\end{figure}
	
	\begin{figure}[h] % [h] forces the figure to be output where it is defined in the code (it suppresses floating)
		\centering
	%	\includegraphics[width=0.8\columnwidth]{rect.jpg} % Example image
		\caption{处理效果}
	\end{figure}
	
	\pagebreak
	\subsubsection{Polygon}
	
	\begin{figure}[h] % [h] forces the figure to be output where it is defined in the code (it suppresses floating)
		\begin{minipage}[t]{0.5\linewidth}
			\centering
	%		\includegraphics[width=0.9\linewidth]{bearpolygon.png}
			\caption{bear}
		\end{minipage}%
		\begin{minipage}[t]{0.5\linewidth}
			\centering
	%		\includegraphics[width=0.9 \linewidth]{girlpolygon.png}
			\caption{girl}
		\end{minipage}
	\end{figure}
	
	\begin{figure}[h] % [h] forces the figure to be output where it is defined in the code (it suppresses floating)
		\centering
	%	\includegraphics[width=0.8\columnwidth]{polygon.jpg} % Example image
		\caption{处理效果}
	\end{figure}
	
	\pagebreak
	
	\subsubsection{Freedraw}
	
	\begin{figure}[h] % [h] forces the figure to be output where it is defined in the code (it suppresses floating)
		\begin{minipage}[t]{0.5\linewidth}
			\centering
	%		\includegraphics[width=0.9\linewidth]{bearfree.png}
			\caption{bear}
		\end{minipage}%
		\begin{minipage}[t]{0.5\linewidth}
			\centering
	%		\includegraphics[width=0.9 \linewidth]{girlfree.png}
			\caption{girl}
		\end{minipage}
	\end{figure}
	
	\begin{figure}[h] % [h] forces the figure to be output where it is defined in the code (it suppresses floating)
		\centering
	%	\includegraphics[width=0.8\columnwidth]{freedraw.jpg} % Example image
		\caption{处理效果}
	\end{figure}
	
	\pagebreak
	\subsubsection{Poisson \& Mix Poisson}
	
	\begin{figure}[h] % [h] forces the figure to be output where it is defined in the code (it suppresses floating)
		\begin{minipage}[t]{0.5\linewidth}
			\centering
	%		\includegraphics[width=0.9\linewidth]{back.jpg}
			\caption{background}
		\end{minipage}%
		\begin{minipage}[t]{0.5\linewidth}
			\centering
	%		\includegraphics[width=0.9 \linewidth]{smile.jpg}
			\caption{sample}
		\end{minipage}
	\end{figure}
	
	\begin{figure}[h] % [h] forces the figure to be output where it is defined in the code (it suppresses floating)
		\centering
	%	\includegraphics[width=0.6\columnwidth]{compare.jpg} % Example image
		\caption{处理效果}
	\end{figure}
	如图,左上为直接复制粘贴,保留前景全部颜色梯度信息;左下为普通Poisson编辑,保留前景全部梯度信息,前景像素颜色与背景作融合;右上为应用混合梯度的Poisson编辑,前景梯度部分保留,效果上比普通Poisson编辑更加“透明”,适合用在水印等场景。
	
	\pagebreak
	\subsection{其他的应用}
	
	\subsubsection{遮盖不必要的信息(如去皱纹)}
	
	原图像:
	
	\begin{figure}[h] % [h] forces the figure to be output where it is defined in the code (it suppresses floating)
		\centering
	%	\includegraphics[width=0.6\columnwidth]{wrinkles.jpg} % Example image
		\caption{抬头纹}
	\end{figure}
	
	处理效果:
	
	\begin{figure}[h] % [h] forces the figure to be output where it is defined in the code (it suppresses floating)
		\centering
	%	\includegraphics[width=0.6\columnwidth]{wrinklesprocess.jpg} % Example image
		\caption{利用脸部其他部位的纹理祛皱}
	\end{figure}
	
	\pagebreak
	
	\subsubsection{恐怖片特效}
	
	原图:
	\begin{figure}[h] % [h] forces the figure to be output where it is defined in the code (it suppresses floating)
		\centering
	%	\includegraphics[width=0.9\columnwidth]{mirror.jpg} % Example image
		\caption{电影《鬼三惊》剧照}
	\end{figure}
	
	镜中人物掩盖:
	\begin{figure}[h] % [h] forces the figure to be output where it is defined in the code (it suppresses floating)
		\centering
	%	\includegraphics[width=0.9\columnwidth]{3.jpg} % Example image
		\caption{处理效果}
	\end{figure}
	
	\pagebreak
	恐怖角色原图:
	\begin{figure}[h] % [h] forces the figure to be output where it is defined in the code (it suppresses floating)
		\centering
	%	\includegraphics[width=0.8\columnwidth]{nuts.jpg} % Example image
		\caption{《招魂》系列中鬼修女形象}
	\end{figure}
	
	处理效果:
	\begin{figure}[h] % [h] forces the figure to be output where it is defined in the code (it suppresses floating)
		\centering
	%		\includegraphics[width=0.8\columnwidth]{4.jpg} % Example image
		\caption{处理效果}
	\end{figure}
	
	
	\pagebreak
	\subsubsection{生成表情包}
	原图:
	\begin{figure}[h] % [h] forces the figure to be output where it is defined in the code (it suppresses floating)
		\centering
	%	\includegraphics[width=0.4\columnwidth]{banana.jpg} % Example image
		\caption{Richard Milos高清图}
	\end{figure}
	
	处理效果:
	\begin{figure}[h] % [h] forces the figure to be output where it is defined in the code (it suppresses floating)
		\centering
	%	\includegraphics[width=0.4\columnwidth]{bananaprocess.jpg} % Example image
		\caption{Richard Milos熊猫头表情包}
	\end{figure}
	
	\pagebreak
	\section{总结}
	本例中忽视了原图像选取范围和图像边界重合即$|N_p|<4$的情况,还可进一步优化。计算上还是采用遍历像素操作的方式进行处理,或许可以对计算方式类似的区域使用矩阵方块操作提高速率
	
	\bibliographystyle{unsrt}
	\bibliography{bibfile}
	
\end{document}