%%%%%%%%%%%%%%%%%%%%%%%%%%%%%%%%%%%%%%%%%
% Wenneker Assignment
% LaTeX Template
% Version 2.0 (12/1/2019)
%
% This template originates from:
% http://www.LaTeXTemplates.com
%
% Authors:
% Vel (vel@LaTeXTemplates.com)
% Frits Wenneker
%
% License:
% CC BY-NC-SA 3.0 (http://creativecommons.org/licenses/by-nc-sa/3.0/)
% 
%%%%%%%%%%%%%%%%%%%%%%%%%%%%%%%%%%%%%%%%%

%----------------------------------------------------------------------------------------
%	PACKAGES AND OTHER DOCUMENT CONFIGURATIONS
%----------------------------------------------------------------------------------------

\documentclass[14pt]{scrartcl} % Font size

%%%%%%%%%%%%%%%%%%%%%%%%%%%%%%%%%%%%%%%%%
% Wenneker Assignment
% Structure Specification File
% Version 2.0 (12/1/2019)
%
% This template originates from:
% http://www.LaTeXTemplates.com
%
% Authors:
% Vel (vel@LaTeXTemplates.com)
% Frits Wenneker
%
% License:
% CC BY-NC-SA 3.0 (http://creativecommons.org/licenses/by-nc-sa/3.0/)
% 
%%%%%%%%%%%%%%%%%%%%%%%%%%%%%%%%%%%%%%%%%

%----------------------------------------------------------------------------------------
%	PACKAGES AND OTHER DOCUMENT CONFIGURATIONS
%----------------------------------------------------------------------------------------

\usepackage{amsmath, amsfonts, amsthm} % Math packages

\usepackage[utf8]{inputenc}

\usepackage[UTF8]{ctex}

\usepackage{listings} % Code listings, with syntax highlighting

\usepackage{subfig}

\usepackage[english]{babel} % English language hyphenation

\usepackage{amssymb}

\usepackage{graphicx} % Required for inserting images
\graphicspath{{Figures/}{./}} % Specifies where to look for included images (trailing slash required)

\usepackage{booktabs} % Required for better horizontal rules in tables

\numberwithin{equation}{section} % Number equations within sections (i.e. 1.1, 1.2, 2.1, 2.2 instead of 1, 2, 3, 4)
\numberwithin{figure}{section} % Number figures within sections (i.e. 1.1, 1.2, 2.1, 2.2 instead of 1, 2, 3, 4)
\numberwithin{table}{section} % Number tables within sections (i.e. 1.1, 1.2, 2.1, 2.2 instead of 1, 2, 3, 4)

\setlength\parindent{0pt} % Removes all indentation from paragraphs

\usepackage{enumitem} % Required for list customisation
\setlist{noitemsep} % No spacing between list items

%----------------------------------------------------------------------------------------
%	DOCUMENT MARGINS
%----------------------------------------------------------------------------------------

\usepackage{geometry} % Required for adjusting page dimensions and margins

\geometry{
	paper=a4paper, % Paper size, change to letterpaper for US letter size
	top=2.5cm, % Top margin
	bottom=3cm, % Bottom margin
	left=3cm, % Left margin
	right=3cm, % Right margin
	headheight=0.75cm, % Header height
	footskip=1.5cm, % Space from the bottom margin to the baseline of the footer
	headsep=0.75cm, % Space from the top margin to the baseline of the header
	%showframe, % Uncomment to show how the type block is set on the page
}

%----------------------------------------------------------------------------------------
%	FONTS
%----------------------------------------------------------------------------------------

\usepackage[utf8]{inputenc} % Required for inputting international characters
\usepackage[T1]{fontenc} % Use 8-bit encoding

\usepackage{fourier} % Use the Adobe Utopia font for the document

%----------------------------------------------------------------------------------------
%	SECTION TITLES
%----------------------------------------------------------------------------------------

\usepackage{sectsty} % Allows customising section commands

\sectionfont{\vspace{6pt}\centering\normalfont\scshape} % \section{} styling
\subsectionfont{\normalfont\bfseries} % \subsection{} styling
\subsubsectionfont{\normalfont\itshape} % \subsubsection{} styling
\paragraphfont{\normalfont\scshape} % \paragraph{} styling

%----------------------------------------------------------------------------------------
%	HEADERS AND FOOTERS
%----------------------------------------------------------------------------------------

\usepackage{scrlayer-scrpage} % Required for customising headers and footers

\ohead*{} % Right header
\ihead*{} % Left header
\chead*{} % Centre header

\ofoot*{} % Right footer
\ifoot*{} % Left footer
\cfoot*{\pagemark} % Centre footer % Include the file specifying the document structure and custom commands

%----------------------------------------------------------------------------------------
%	TITLE SECTION
%----------------------------------------------------------------------------------------

\title{	
	\normalfont\normalsize
	%\vspace{25pt} % Whitespace
	\rule{\linewidth}{0.5pt}\\ % Thin top horizontal rule
	\vspace{20pt} % Whitespace
	{\huge 实验六	Mass Spring}\\ % The assignment title
	\vspace{12pt} % Whitespace
	\rule{\linewidth}{2pt}\\ % Thick bottom horizontal rule
	\vspace{12pt} % Whitespace
}

\author{\LARGE ID: 58		陈文博} % Your name

\date{\normalsize\today} % Today's date (\today) or a custom date

\begin{document}

\maketitle % Print the title

%----------------------------------------------------------------------------------------
%	FIGURE EXAMPLE
%----------------------------------------------------------------------------------------

\section{实验要求}

%\begin{figure}[h] % [h] forces the figure to be output where it is defined in the code (it suppresses floating)
%	\centering
%	%\includegraphics[width=0.5\columnwidth]{swallow.jpg} % Example image
%	\caption{European swallow.}
%\end{figure}

\begin{itemize}
	\item[*] 在给定的网格框架上完成作业,实现
	\begin{itemize}
		\item 质点弹簧仿真模型的欧拉隐式方法
		\item 质点弹簧仿真模型的加速模拟方法
	\end{itemize}
	\item[*] 学习使用 Tetgen 库生成四面体剖分
\end{itemize}

%------------------------------------------------

\section{开发环境}

\textbf{IDE}:Microsoft Visual Studio 2019 community\\
\textbf{CMake}:3.16.3\\
\textbf{Qt}:5.14.1\\
\textbf{Eigen}:3.3.7\\
\textbf{Assimp}:5.0.1\\
\textbf{tinyxml2}:8.0.0\\
\textbf{Others}

\section{界面描述}
\begin{figure}[h] % [h] forces the figure to be output where it is defined in the code (it suppresses floating)
	\centering
	%\includegraphics[width=1.1\columnwidth]{menu.png} % Example image
	\caption{界面描述}
\end{figure}

%----------------------------------------------------------------------------------------
%	TEXT EXAMPLE
%----------------------------------------------------------------------------------------
\pagebreak
\section{算法原理}

\subsection{弹簧质点模型}
一个弹簧质点系统就是由节点及节点之间的边所构成的图,也就是网格。网格图的每个顶点看为一个质点,每条边看为一根弹簧。网格可以是二维网格,用于模拟布料、纸张等物体;也可以是三维体网格,用于模拟体物体。

\subsection{隐式欧拉方法}
已知前$n$帧的信息,设位移为$\boldsymbol x$,速度为$\boldsymbol v$,时间步长为$h$,现计算第$n+1$帧信息\\

第$n+1$帧信息与第$n$帧信息满足以下关系:
\begin{equation}
\begin{aligned}
\boldsymbol x_{n+1}&=\boldsymbol x_n+h\boldsymbol  v_{n+1}\\
\boldsymbol v_{n+1}&=\boldsymbol v_n+h\boldsymbol M^{-1}(\boldsymbol f_{int}(t_{n+1})+\boldsymbol f_{ext})
\end{aligned}
\end{equation}

记
\begin{equation}
\boldsymbol y=\boldsymbol x_n+h\boldsymbol v_n+h^2\boldsymbol M^{-1}\boldsymbol f_{ext}
\end{equation}

原问题转化为求解关于$\boldsymbol x$的方程:

\begin{equation}
\boldsymbol g(\boldsymbol x)=\boldsymbol M(\boldsymbol x-\boldsymbol y)-h^2\boldsymbol f_{int}(\boldsymbol x)=0
\end{equation}

利用牛顿法求解该方程,主要迭代步骤:
\begin{equation}
\boldsymbol x^{(k+1)}=\boldsymbol x^{(k)}-(\nabla \boldsymbol g(\boldsymbol x^{(k)}))^{-1}\boldsymbol g(\boldsymbol x^{(k)})
\end{equation}

迭代初值可选$\boldsymbol x^{(0)}=\boldsymbol y$\\
迭代得到位移$\boldsymbol x$后更新速度:$\boldsymbol v_{n+1}=(\boldsymbol x_{n+1}-\boldsymbol x_n)/h$

\pagebreak
\subsection{弹力求导问题}
对于单个弹簧(端点为$\boldsymbol  x_1$,$\boldsymbol  x_2$),劲度系数为$k$,原长为$l$,有:
\begin{equation}
\begin{aligned}
\boldsymbol f_1(\boldsymbol x_1,\boldsymbol x_2)&=k(||\boldsymbol x_1-\boldsymbol x_2||-l)\frac{\boldsymbol x_2-\boldsymbol x_1}{||\boldsymbol x_1-\boldsymbol x_2||}\\
\boldsymbol f_2(\boldsymbol x_1,\boldsymbol x_2)&=-\boldsymbol f_1(\boldsymbol x_1,\boldsymbol x_2)
\end{aligned}
\end{equation}

对
\begin{equation}
\boldsymbol h(\boldsymbol x)=k(||\boldsymbol x||-l)\frac{-\boldsymbol x}{||\boldsymbol x||}
\end{equation}

求导有
\begin{equation}
\frac{ d  \boldsymbol h}{d \boldsymbol x} = k(\frac{l}{||\boldsymbol x||}-1)\boldsymbol I-kl||\boldsymbol x||^{-3}\boldsymbol x \boldsymbol x^T
\end{equation}

代入弹力公式得:
\begin{equation}
\frac{\partial  \boldsymbol f_1}{\partial \boldsymbol x_1} =\frac{\partial  \boldsymbol h(\boldsymbol x_1-\boldsymbol x_2)}{\partial \boldsymbol x_1}=k(\frac{l}{||\boldsymbol r||}-1)\boldsymbol I-kl||\boldsymbol r||^{-3}\boldsymbol r \boldsymbol r^T
\end{equation}
\begin{equation}
\frac{\partial  \boldsymbol f_1}{\partial \boldsymbol x_2}=-\frac{\partial  \boldsymbol f_1}{\partial \boldsymbol x_1},
\frac{\partial  \boldsymbol f_2}{\partial \boldsymbol x_1}=-\frac{\partial  \boldsymbol f_1}{\partial \boldsymbol x_1},
\frac{\partial  \boldsymbol f_2}{\partial \boldsymbol x_2}=\frac{\partial  \boldsymbol f_1}{\partial \boldsymbol x_1}
\end{equation}
其中,$\boldsymbol r=\boldsymbol  x_1-\boldsymbol x_2$,$\boldsymbol I$为单位阵\\

\subsection{加速方法\cite{Fast}}
对于弹簧的弹性势能可以描述为一个最小化问题:
\begin{equation}
\frac{1}{2}k(\|\boldsymbol  p_1-\boldsymbol p_2\|-r)^2=\frac{1}{2}k\min\limits_{\|\boldsymbol d\|=r}\|\boldsymbol p_1-\boldsymbol p_2-\boldsymbol  d\|^2
\end{equation}
原问题转化为:
\begin{equation}
\boldsymbol x_{n+1}=\min\limits_{x,\boldsymbol d\in \boldsymbol U}\frac{1}{2} \boldsymbol x^T(\boldsymbol M+h^2\boldsymbol L)\boldsymbol x-h^2\boldsymbol x^T\boldsymbol J\boldsymbol d-\boldsymbol x^T\boldsymbol M\boldsymbol y
\end{equation}

其中,$\boldsymbol U=\{\boldsymbol d=(\boldsymbol d_1,\boldsymbol d_2,\cdots,\boldsymbol d_s),\boldsymbol d_s\in R^3,\|\boldsymbol d_i\|=l_i\}$($l_i$为第$i$个弹簧的原长)\\

矩阵$\boldsymbol L\in R^{3m\times 3m}$,$\boldsymbol J\in R^{3m\times 3s}$定义为:
\begin{equation}
\begin{aligned}
\boldsymbol L=\left(\sum\limits_{i=1}^sk_i\boldsymbol A_i\boldsymbol A_i^T\right)\otimes\boldsymbol I_3,
\boldsymbol J=\left(\sum\limits_{i=1}^sk_i\boldsymbol A_i\boldsymbol S_i^T\right)\otimes \boldsymbol I_3
\end{aligned}
\end{equation}

其中$\boldsymbol A_i$为第$i$根弹簧的关联向量(如第$i$根弹簧端点序号为$i_1$和$i_2$,则$\boldsymbol A_{i,i_1}=1$,$\boldsymbol A_{i,i_2}=-1$,其余元素为0),$\boldsymbol S_i$为第$i$条弹簧的指示矩阵,$S_{i,j}=\delta_{i,j}$\\

从而可以对 $\boldsymbol x$,$\boldsymbol d$ 迭代优化求得该优化问题的解:

\begin{equation}
\begin{aligned}
&\mathrm{optimize}\ \boldsymbol x:(\boldsymbol M+h^2\boldsymbol L)\boldsymbol x=h^2\boldsymbol J\boldsymbol d+\boldsymbol M\boldsymbol y\\
&\mathrm{optimize}\ \boldsymbol d:\boldsymbol d_i=l_i\frac{\boldsymbol p_{i_1}-\boldsymbol p_{i_2}}{\|\boldsymbol p_{i_1}-\boldsymbol p_{i_2}\|}
\end{aligned}
\end{equation}
重复迭代直至收敛

\subsection{位移约束的处理}
将所有n个质点的坐标列为列向量 $x\in R^{3n}$,将所有 m 个自由质点坐标(无约束坐标)列为列向量 $x_f\in R^{3m}$,则两者关系:
\begin{equation}
\begin{aligned}
\boldsymbol x_f&=\boldsymbol K\boldsymbol x,\\  \boldsymbol x&=\boldsymbol K^T\boldsymbol x_f+\boldsymbol b\
\end{aligned}
\end{equation}

其中 $\boldsymbol K\in R^{3m\times 3n}$ 为单位阵删去约束坐标序号对应行所得的稀疏矩阵,$\boldsymbol b$ 为与约束位移有关的向量,计算为 $\boldsymbol b=\boldsymbol x-\boldsymbol K^T\boldsymbol K\boldsymbol x$, 若约束为固定质点则 $\boldsymbol b$ 为常量。由此我们将原本的关于 $\boldsymbol x$ 的优化问题转化为对 $\boldsymbol x_f$ 的优化问题:欧拉隐式方法中求解方程为:
\begin{equation}
\begin{aligned}
\boldsymbol g_1(\boldsymbol x_f) &= \boldsymbol K(\boldsymbol M(\boldsymbol x-\boldsymbol y) -h^2\boldsymbol f_{int}(\boldsymbol x)) = 0,\\
\nabla_{\boldsymbol x_f} \boldsymbol g_1(\boldsymbol x_f) &= \boldsymbol K\nabla_{\boldsymbol x} \boldsymbol g(\boldsymbol x)\boldsymbol K^T,\\
\end{aligned}
\end{equation}
加速方法中优化问题中 $\boldsymbol x$ 迭代步骤转化为求解关于 $x_f$ 的方程:
\begin{equation}
\boldsymbol K(\boldsymbol M+h^2\boldsymbol L)\boldsymbol K^T\boldsymbol x_f=\boldsymbol K(h^2\boldsymbol J \boldsymbol d+ \boldsymbol M \boldsymbol y-(\boldsymbol M+h^2\boldsymbol L)\boldsymbol b)
\end{equation}

%------------------------------------------------
\pagebreak
\section{设计难点与解决}

\subsection{隐式欧拉收敛性问题}
使用普通隐式欧拉方法时,当stiff设置比较大时容易发生顶点的大幅度偏移直至无法复原,检查每次迭代的增量发现计算并没有往最优解方向收敛,之后尝试将时间步长h调小,可以一定程度上缓解,而使用加速方法进行迭代则没有这种情况发生。

\subsection{使用带约束的加速方法时设置固定点与实际固定点不符}
进行位移约束时需要构造一个单位阵删去约束坐标序号对应行所得的稀疏矩阵,由于采用循环删除,循环过程中矩阵行数发生变化而删除索引未作相应改变导致出错。

\subsection{收敛条件}
计算每次迭代所有点增量的平均值,当平均增量小于1e-4时认为收敛,停止迭代,更新速度和位移。

\subsection{矩阵预分解}
使用Eigen库的LDLT稀疏矩阵分解

\pagebreak

\section{实验效果}

\subsection{布料仿真效果}
\begin{figure}[h] % [h] forces the figure to be output where it is defined in the code (it suppresses floating)
	\centering
	%\includegraphics[width=0.5\columnwidth]{dense1.png} % Example image
	\caption{稠密布料}
\end{figure}
\begin{figure}[h] % [h] forces the figure to be output where it is defined in the code (it suppresses floating)
	\centering
	%\includegraphics[width=0.5\columnwidth]{sparse.png} % Example image
	\caption{稀疏布料}
\end{figure}

详见demo1.gif和demo7.gif
\pagebreak
\subsection{固体质量块仿真效果}
\begin{figure}[h] % [h] forces the figure to be output where it is defined in the code (it suppresses floating)
	\centering
	%\includegraphics[width=0.5\columnwidth]{cube.png} % Example image
	\caption{cube}
\end{figure}
\begin{figure}[h] % [h] forces the figure to be output where it is defined in the code (it suppresses floating)
	\centering
	%\includegraphics[width=0.5\columnwidth]{cube1.png} % Example image
	\caption{劲度系数较小时的变形}
\end{figure}
详见demo2.gif和demo6.gif

\pagebreak
\subsection{“布脸”}
\begin{figure}[h] % [h] forces the figure to be output where it is defined in the code (it suppresses floating)
	\centering
	%\includegraphics[width=0.5\columnwidth]{face1.png} % Example image
	\caption{初态}
\end{figure}
\begin{figure}[h] % [h] forces the figure to be output where it is defined in the code (it suppresses floating)
	\centering
	%\includegraphics[width=0.5\columnwidth]{face2.png} % Example image
	\caption{末态}
\end{figure}
详见demo3.gifh

\pagebreak
\subsection{“变形怪”}
\begin{figure}[h] % [h] forces the figure to be output where it is defined in the code (it suppresses floating)
	\centering
	%\includegraphics[width=0.5\columnwidth]{dito.png} % Example image
	\caption{初态}
\end{figure}
\begin{figure}[h] % [h] forces the figure to be output where it is defined in the code (it suppresses floating)
	\centering
	%\includegraphics[width=0.5\columnwidth]{dito1.png} % Example image
	\caption{变形}
\end{figure}
详见demo4.gif

\pagebreak
\subsection{劲度系数测试}
\begin{figure}[h] % [h] forces the figure to be output where it is defined in the code (it suppresses floating)
	\centering
	%\includegraphics[width=0.5\columnwidth]{stiff.png} % Example image
	\caption{stiff}
\end{figure}
详见demo5.gif

\subsection{对某一点施加外力}

\begin{figure}[h] % [h] forces the figure to be output where it is defined in the code (it suppresses floating)
	\centering
	%\includegraphics[width=0.5\columnwidth]{fext.png} % Example image
	\caption{Fext}
\end{figure}
详见demo8.gif

\pagebreak
\section{总结}
本次实验提供的资料比较全面,实现起来不会有很多疑惑,实验效果也比较直观生动,也希望今后能有机会做更多有趣的、有挑战性的仿真工作

\bibliographystyle{unsrt}
\bibliography{bibfile}

\end{document}