%%%%%%%%%%%%%%%%%%%%%%%%%%%%%%%%%%%%%%%%%
% Wenneker Assignment
% LaTeX Template
% Version 2.0 (12/1/2019)
%
% This template originates from:
% http://www.LaTeXTemplates.com
%
% Authors:
% Vel (vel@LaTeXTemplates.com)
% Frits Wenneker
%
% License:
% CC BY-NC-SA 3.0 (http://creativecommons.org/licenses/by-nc-sa/3.0/)
% 
%%%%%%%%%%%%%%%%%%%%%%%%%%%%%%%%%%%%%%%%%

%----------------------------------------------------------------------------------------
%	PACKAGES AND OTHER DOCUMENT CONFIGURATIONS
%----------------------------------------------------------------------------------------

\documentclass[14pt]{scrartcl} % Font size

%%%%%%%%%%%%%%%%%%%%%%%%%%%%%%%%%%%%%%%%%
% Wenneker Assignment
% Structure Specification File
% Version 2.0 (12/1/2019)
%
% This template originates from:
% http://www.LaTeXTemplates.com
%
% Authors:
% Vel (vel@LaTeXTemplates.com)
% Frits Wenneker
%
% License:
% CC BY-NC-SA 3.0 (http://creativecommons.org/licenses/by-nc-sa/3.0/)
% 
%%%%%%%%%%%%%%%%%%%%%%%%%%%%%%%%%%%%%%%%%

%----------------------------------------------------------------------------------------
%	PACKAGES AND OTHER DOCUMENT CONFIGURATIONS
%----------------------------------------------------------------------------------------

\usepackage{amsmath, amsfonts, amsthm} % Math packages

\usepackage[utf8]{inputenc}

\usepackage[UTF8]{ctex}

\usepackage{listings} % Code listings, with syntax highlighting

\usepackage{subfig}

\usepackage[english]{babel} % English language hyphenation

\usepackage{amssymb}

\usepackage{graphicx} % Required for inserting images
\graphicspath{{Figures/}{./}} % Specifies where to look for included images (trailing slash required)

\usepackage{booktabs} % Required for better horizontal rules in tables

\numberwithin{equation}{section} % Number equations within sections (i.e. 1.1, 1.2, 2.1, 2.2 instead of 1, 2, 3, 4)
\numberwithin{figure}{section} % Number figures within sections (i.e. 1.1, 1.2, 2.1, 2.2 instead of 1, 2, 3, 4)
\numberwithin{table}{section} % Number tables within sections (i.e. 1.1, 1.2, 2.1, 2.2 instead of 1, 2, 3, 4)

\setlength\parindent{0pt} % Removes all indentation from paragraphs

\usepackage{enumitem} % Required for list customisation
\setlist{noitemsep} % No spacing between list items

%----------------------------------------------------------------------------------------
%	DOCUMENT MARGINS
%----------------------------------------------------------------------------------------

\usepackage{geometry} % Required for adjusting page dimensions and margins

\geometry{
	paper=a4paper, % Paper size, change to letterpaper for US letter size
	top=2.5cm, % Top margin
	bottom=3cm, % Bottom margin
	left=3cm, % Left margin
	right=3cm, % Right margin
	headheight=0.75cm, % Header height
	footskip=1.5cm, % Space from the bottom margin to the baseline of the footer
	headsep=0.75cm, % Space from the top margin to the baseline of the header
	%showframe, % Uncomment to show how the type block is set on the page
}

%----------------------------------------------------------------------------------------
%	FONTS
%----------------------------------------------------------------------------------------

\usepackage[utf8]{inputenc} % Required for inputting international characters
\usepackage[T1]{fontenc} % Use 8-bit encoding

\usepackage{fourier} % Use the Adobe Utopia font for the document

%----------------------------------------------------------------------------------------
%	SECTION TITLES
%----------------------------------------------------------------------------------------

\usepackage{sectsty} % Allows customising section commands

\sectionfont{\vspace{6pt}\centering\normalfont\scshape} % \section{} styling
\subsectionfont{\normalfont\bfseries} % \subsection{} styling
\subsubsectionfont{\normalfont\itshape} % \subsubsection{} styling
\paragraphfont{\normalfont\scshape} % \paragraph{} styling

%----------------------------------------------------------------------------------------
%	HEADERS AND FOOTERS
%----------------------------------------------------------------------------------------

\usepackage{scrlayer-scrpage} % Required for customising headers and footers

\ohead*{} % Right header
\ihead*{} % Left header
\chead*{} % Centre header

\ofoot*{} % Right footer
\ifoot*{} % Left footer
\cfoot*{\pagemark} % Centre footer % Include the file specifying the document structure and custom commands

%----------------------------------------------------------------------------------------
%	TITLE SECTION
%----------------------------------------------------------------------------------------

\title{	
	\normalfont\normalsize
	%\vspace{25pt} % Whitespace
	\rule{\linewidth}{0.5pt}\\ % Thin top horizontal rule
	\vspace{20pt} % Whitespace
	{\huge 实验四	\\Minimal Surface\\\&\\Mesh Parameterization}\\ % The assignment title
	\vspace{12pt} % Whitespace
	\rule{\linewidth}{2pt}\\ % Thick bottom horizontal rule
	\vspace{12pt} % Whitespace
}

\author{\LARGE ID: 58		陈文博} % Your name

\date{\normalsize\today} % Today's date (\today) or a custom date

\begin{document}
	
	\maketitle % Print the title
	
	%----------------------------------------------------------------------------------------
	%	FIGURE EXAMPLE
	%----------------------------------------------------------------------------------------
	
	\section{实验要求}
	
	%\begin{figure}[h] % [h] forces the figure to be output where it is defined in the code (it suppresses floating)
	%	\centering
	%	%\includegraphics[width=0.5\columnwidth]{swallow.jpg} % Example image
	%	\caption{European swallow.}
	%\end{figure}
	
	\begin{itemize}
		\item[*] 初步理解*.obj数据(*.obj, *.mtl)
		\begin{itemize}
			\item 安装MeshLab查看三维数据文件
		\end{itemize}
		\item[*] 学习网格的数据结构及操作
		\begin{itemize}
			\item 使用MeshFrame框架
			\item 寻找非封闭网格曲面的边界	
		\end{itemize}
		\item[*] 实现极小曲面与网格参数化
		\begin{itemize}
			\item 极小曲面:边界固定,求解方程组
			\item 参数化:边界映射到平面,求解方程组	
		\end{itemize}
		\item[*] 巩固使用Eigen库求解稀疏线性方程组
		
		
	\end{itemize}
	
	%------------------------------------------------
	
	\section{开发环境}
	
	\textbf{IDE}:Microsoft Visual Studio 2019 community\\
	\textbf{CMake}:3.16.3\\
	\textbf{Qt}:5.14.1\\
	\textbf{Eigen}:3.3.7\\
	\textbf{Assimp}:5.0.1\\
	\textbf{tinyxml2}:8.0.0\\
	\textbf{Others}
	
	
	%----------------------------------------------------------------------------------------
	%	TEXT EXAMPLE
	%----------------------------------------------------------------------------------------
	\pagebreak
	\section{算法原理}
	
	\subsection{极小曲面Minimal Surface}
	
	\textbf{定义:}平均曲率处处为0的曲面\\
	\textbf{算法:}\\
	由于曲面的平均曲率处处为0,即:
	\begin{equation}
		H(v_i)=0,\forall i
	\end{equation}
	\begin{figure}[h] % [h] forces the figure to be output where it is defined in the code (it suppresses floating)
		\centering
		%\includegraphics[width=0.8\columnwidth]{1.jpg} % Example image
		\caption{曲率图示}
	\end{figure}
	
	如图,有:
	\begin{equation}
		\lim\limits_{len(y)\rightarrow 0}\frac{1}{len(\gamma)}\int_{v\in \gamma}(v_i-v)ds=H(v_i)\boldsymbol{n}_i
	\end{equation}
	
	微分坐标为:
	\begin{equation}
		\delta_i=v_i-\frac{1}{d_i}\sum\limits_{v\in N(i)} v =\frac{1}{d_i}\sum\limits_{v\in N(i)} (v_i-v)=0
	\end{equation}
	
	固定边界点,通过求解稀疏方程组可以得到最小平面的顶点坐标
	\pagebreak
	
	\subsection{网格参数化Mesh Parameterization}
	
	将网格边界映射到凸多边形(如单位圆、正方形)上,对每个内部点生成关于其领域点$N(i)$的重心坐标$(\lambda_{i1},\lambda_{i2},\cdots,\lambda_{id_i}$,利用重心坐标得到:
	\begin{equation}
		v_i-\sum\limits_{j\in N(i)}\lambda_{ij}v_j=0,i=1,\cdots,n
	\end{equation}
	
	其中$n$为内部点数量,通过求解该稀疏方程组,可得到曲面参数化后的坐标,对于不同的重心坐标求取方法,得到的参数化网格也不同。利用添加参数化后的纹理坐标连接纹理图像,可实现纹理映射,如下图,不同重心坐标纹理映射效果也不同:
	\begin{figure}[h] % [h] forces the figure to be output where it is defined in the code (it suppresses floating)
		\centering
		%\includegraphics[width=1\columnwidth]{begin.png} % Example image
		\caption{纹理映射}
	\end{figure}
	
	\subsubsection{Uniform}
	均匀重心坐标,只考虑内部点的度而不考虑与其邻接点的关系,即取重心坐标:
	\begin{equation}
		\lambda_{ij}=\frac{1}{d_i},d_i=|N(i)|,j\in N(i)
	\end{equation}
	每个邻接点权重相等
	
	\subsubsection{Cotangent}
	余切重心坐标,考虑了内部点与邻接点的位置关系,如下图所示,设某一内部点$v$顺次连接的相邻点为$v_{i-1}$,$v_i$,$v_{i+1}$,设$\beta_{i-1}=\angle{vv_{i-1}v_i}$和$\gamma_i=\angle{vv_{i+1}v_i}$
	\begin{figure}[h] % [h] forces the figure to be output where it is defined in the code (it suppresses floating)
		\centering
		%\includegraphics[width=0.4\columnwidth]{2.png} % Example image
		\caption{discrete harmonic coordinates}
	\end{figure}
	
	重心坐标如下:
	\begin{equation}
		\lambda_j=\frac{w_j}{\sum\limits_{j\in N(v)}w_j}
	\end{equation}
	\begin{equation}
		w_j=\cot(\beta_{i-1})+\cot(\gamma_i),j\in N(v)
	\end{equation}
	
	
	\pagebreak
	\section{设计难点与解决}
	
	\subsection{UEngine框架的使用}
	
	作业提供的UEngine框架提供了作业需要的模型加载渲染显示等基础功能,作业的极小曲面和网格参数化算法通过MinSurf和Parameterization两个类实现,同时在attribute中添加多个按钮以方便交互
	%------------------------------------------------
	
	\subsection{度为2的边界顶点的问题}
	当网格中出现度为2的边界顶点时,若使用正方形作为参数化的固定边界模式可能会发生崩溃,例如测试模型中的bunny\_head.obj(如下图)
	\begin{figure}[h] % [h] forces the figure to be output where it is defined in the code (it suppresses floating)
		\centering
		%\includegraphics[width=0.5\columnwidth]{3.png} % Example image
		\caption{bunny\_head.obj中的特殊顶点 }
	\end{figure}
	
	当展开到正方形边界时,可能会出现以下情况:
	\begin{figure}[h] % [h] forces the figure to be output where it is defined in the code (it suppresses floating)
		\centering
		%\includegraphics[width=0.5\columnwidth]{4.png} % Example image
		\caption{特殊情况 }
	\end{figure}
	
	这种情况将发生错误,实际编程中仍能够渲染出纹理贴图,但查看参数化网格图时会出现报错,由于时间原因暂无好的解决方法。
	
	\pagebreak
	
	\section{实验效果}
	
	\subsection{极小曲面Minimal Surface}
	\begin{table}[h] % [h] forces the table to be output where it is defined in the code (it suppresses floating)
		\centering % Centre the table
		\begin{tabular}{l l l l}
			\toprule
			\centering
			\textbf{原曲面} & \textbf{原网格} & \textbf{极小曲面} &\textbf{极小曲面网格}\\
			\midrule
			%%%%%%%%%%%%%%%%%%%%%%%%%%%%%%%%
			\begin{minipage}[t]{0.2\linewidth}
				\centering
				%\includegraphics[width=1\linewidth]{ball.png}
			\end{minipage}&
			\begin{minipage}[t]{0.2\linewidth}
				\centering
				%\includegraphics[width=1\linewidth]{ball_mesh.png}
			\end{minipage}&
			\begin{minipage}[t]{0.2\linewidth}
				\centering
				%\includegraphics[width=1\linewidth]{ball_minisuef.png}
			\end{minipage}&
			\begin{minipage}[t]{0.2\linewidth}
				\centering
				%\includegraphics[width=1\linewidth]{ball_minisuef_mesh.png}
			\end{minipage}\\
			%%%%%%%%%%%%%%%%%%%%%%%%%%%%%%%%%%%%%%%%%%%%%
			\begin{minipage}[t]{0.2\linewidth}
				\centering
				%\includegraphics[width=1\linewidth]{bunny.png}
			\end{minipage}&
			\begin{minipage}[t]{0.2\linewidth}
				\centering
				%\includegraphics[width=1\linewidth]{bunny_mesh.png}
			\end{minipage}&
			\begin{minipage}[t]{0.2\linewidth}
				\centering
				%\includegraphics[width=1\linewidth]{bunny_minisuef.png}
			\end{minipage}&
			\begin{minipage}[t]{0.2\linewidth}
				\centering
				%\includegraphics[width=1\linewidth]{bunny_minisuef_mesh.png}
			\end{minipage}\\
			
			%%%%%%%%%%%%%%%%%%%%%%%%%%%%%%%%%%%%%%%%%%%%%
			\begin{minipage}[t]{0.2\linewidth}
				\centering
				%\includegraphics[width=1\linewidth]{cat.png}
			\end{minipage}&
			\begin{minipage}[t]{0.2\linewidth}
				\centering
				%\includegraphics[width=1\linewidth]{cat_mesh.png}
			\end{minipage}&
			\begin{minipage}[t]{0.2\linewidth}
				\centering
				%\includegraphics[width=1\linewidth]{cat_minisuef.png}
			\end{minipage}&
			\begin{minipage}[t]{0.2\linewidth}
				\centering
				%\includegraphics[width=1\linewidth]{cat_minisuef_mesh.png}
			\end{minipage}\\
			
			%%%%%%%%%%%%%%%%%%%%%%%%%%%%%%%%%%%%%%%%%%%%%
			\begin{minipage}[t]{0.2\linewidth}
				\centering
				%\includegraphics[width=1\linewidth]{david.png}
			\end{minipage}&
			\begin{minipage}[t]{0.2\linewidth}
				\centering
				%\includegraphics[width=1\linewidth]{david_mesh.png}
			\end{minipage}&
			\begin{minipage}[t]{0.2\linewidth}
				\centering
				%\includegraphics[width=1\linewidth]{david_minisuef.png}
			\end{minipage}&
			\begin{minipage}[t]{0.2\linewidth}
				\centering
				%\includegraphics[width=1\linewidth]{david_minisuef_mesh.png}
			\end{minipage}\\
			
			%%%%%%%%%%%%%%%%%%%%%%%%%%%%%%%%%%%%%%%%%%%%%
			\begin{minipage}[t]{0.2\linewidth}
				\centering
				%\includegraphics[width=1\linewidth]{face.png}
			\end{minipage}&
			\begin{minipage}[t]{0.2\linewidth}
				\centering
				%\includegraphics[width=1\linewidth]{face_mesh.png}
			\end{minipage}&
			\begin{minipage}[t]{0.2\linewidth}
				\centering
				%\includegraphics[width=1\linewidth]{face_minisuef.png}
			\end{minipage}&
			\begin{minipage}[t]{0.2\linewidth}
				\centering
				%\includegraphics[width=1\linewidth]{face_minisuef_mesh.png}
			\end{minipage}\\
			
			
		\end{tabular}
		\caption{极小曲面}
	\end{table}
	
	
	\pagebreak
	\subsection{网格参数化与纹理映射}
	\subsubsection{Uniform权重\&单位圆边界}
	\begin{table}[h] % [h] forces the table to be output where it is defined in the code (it suppresses floating)
		\centering % Centre the table
		\begin{tabular}{l l l l}
			\toprule
			\centering
			\textbf{原曲面} & \textbf{原网格} & \textbf{参数化网格} &\textbf{纹理映射}\\
			\midrule
			%%%%%%%%%%%%%%%%%%%%%%%%%%%%%%%%
			\begin{minipage}[t]{0.2\linewidth}
				\centering
				%\includegraphics[width=1\linewidth]{ball.png}
			\end{minipage}&
			\begin{minipage}[t]{0.2\linewidth}
				\centering
				%\includegraphics[width=1\linewidth]{ball_mesh.png}
			\end{minipage}&
			\begin{minipage}[t]{0.2\linewidth}
				\centering
				%\includegraphics[width=1\linewidth]{ball_circle_uniform_mesh.png}
			\end{minipage}&
			\begin{minipage}[t]{0.2\linewidth}
				\centering
				%\includegraphics[width=1\linewidth]{ball_circle_uniform.png}
			\end{minipage}\\
			%%%%%%%%%%%%%%%%%%%%%%%%%%%%%%%%%%%%%%%%%%%%%
			\begin{minipage}[t]{0.2\linewidth}
				\centering
				%\includegraphics[width=1\linewidth]{bunny.png}
			\end{minipage}&
			\begin{minipage}[t]{0.2\linewidth}
				\centering
				%\includegraphics[width=1\linewidth]{bunny_mesh.png}
			\end{minipage}&
			\begin{minipage}[t]{0.2\linewidth}
				\centering
				%\includegraphics[width=1\linewidth]{bunny_circle_uniform_mesh.png}
			\end{minipage}&
			\begin{minipage}[t]{0.2\linewidth}
				\centering
				%\includegraphics[width=1\linewidth]{bunny_circle_uniform.png}
			\end{minipage}\\
			
			%%%%%%%%%%%%%%%%%%%%%%%%%%%%%%%%%%%%%%%%%%%%%
			\begin{minipage}[t]{0.2\linewidth}
				\centering
				%\includegraphics[width=1\linewidth]{cat.png}
			\end{minipage}&
			\begin{minipage}[t]{0.2\linewidth}
				\centering
				%\includegraphics[width=1\linewidth]{cat_mesh.png}
			\end{minipage}&
			\begin{minipage}[t]{0.2\linewidth}
				\centering
				%\includegraphics[width=1\linewidth]{cat_circle_uniform_mesh.png}
			\end{minipage}&
			\begin{minipage}[t]{0.2\linewidth}
				\centering
				%\includegraphics[width=1\linewidth]{cat_circle_uniform.png}
			\end{minipage}\\
			
			%%%%%%%%%%%%%%%%%%%%%%%%%%%%%%%%%%%%%%%%%%%%%
			\begin{minipage}[t]{0.2\linewidth}
				\centering
				%\includegraphics[width=1\linewidth]{david.png}
			\end{minipage}&
			\begin{minipage}[t]{0.2\linewidth}
				\centering
				%\includegraphics[width=1\linewidth]{david_mesh.png}
			\end{minipage}&
			\begin{minipage}[t]{0.2\linewidth}
				\centering
				%\includegraphics[width=1\linewidth]{david_circle_uniform_mesh.png}
			\end{minipage}&
			\begin{minipage}[t]{0.2\linewidth}
				\centering
				%\includegraphics[width=1\linewidth]{david_circle_uniform.png}
			\end{minipage}\\
			
			%%%%%%%%%%%%%%%%%%%%%%%%%%%%%%%%%%%%%%%%%%%%%
			\begin{minipage}[t]{0.2\linewidth}
				\centering
				%\includegraphics[width=1\linewidth]{face.png}
			\end{minipage}&
			\begin{minipage}[t]{0.2\linewidth}
				\centering
				%\includegraphics[width=1\linewidth]{face_mesh.png}
			\end{minipage}&
			\begin{minipage}[t]{0.2\linewidth}
				\centering
				%\includegraphics[width=1\linewidth]{face_circle_uniform_mesh.png}
			\end{minipage}&
			\begin{minipage}[t]{0.2\linewidth}
				\centering
				%\includegraphics[width=1\linewidth]{face_circle_uniform.png}
			\end{minipage}\\
			
			
		\end{tabular}
		\caption{Uniform权重,单位圆边界的网格参数化与纹理映射}
	\end{table}			
	
	
	\pagebreak
	
	\subsubsection{Uniform权重\&正方形边界}
	\begin{table}[h] % [h] forces the table to be output where it is defined in the code (it suppresses floating)
		\centering % Centre the table
		\begin{tabular}{l l l l}
			\toprule
			\centering
			\textbf{原曲面} & \textbf{原网格} & \textbf{参数化网格} &\textbf{纹理映射}\\
			\midrule
			%%%%%%%%%%%%%%%%%%%%%%%%%%%%%%%%
			\begin{minipage}[t]{0.2\linewidth}
				\centering
				%\includegraphics[width=1\linewidth]{ball.png}
			\end{minipage}&
			\begin{minipage}[t]{0.2\linewidth}
				\centering
				%\includegraphics[width=1\linewidth]{ball_mesh.png}
			\end{minipage}&
			\begin{minipage}[t]{0.2\linewidth}
				\centering
				%\includegraphics[width=1\linewidth]{ball_square_uniform_mesh.png}
			\end{minipage}&
			\begin{minipage}[t]{0.2\linewidth}
				\centering
				%\includegraphics[width=1\linewidth]{ball_square_uniform.png}
			\end{minipage}\\
			%%%%%%%%%%%%%%%%%%%%%%%%%%%%%%%%%%%%%%%%%%%%%
			\begin{minipage}[t]{0.2\linewidth}
				\centering
				%\includegraphics[width=1\linewidth]{bunny.png}
			\end{minipage}&
			\begin{minipage}[t]{0.2\linewidth}
				\centering
				%\includegraphics[width=1\linewidth]{bunny_mesh.png}
			\end{minipage}&
			\begin{minipage}[t]{0.2\linewidth}
				\centering
				%\includegraphics[width=1\linewidth]{bunny_square_uniform_mesh.png}
			\end{minipage}&
			\begin{minipage}[t]{0.2\linewidth}
				\centering
				%\includegraphics[width=1\linewidth]{bunny_square_uniform.png}
			\end{minipage}\\
			
			%%%%%%%%%%%%%%%%%%%%%%%%%%%%%%%%%%%%%%%%%%%%%
			\begin{minipage}[t]{0.2\linewidth}
				\centering
				%\includegraphics[width=1\linewidth]{cat.png}
			\end{minipage}&
			\begin{minipage}[t]{0.2\linewidth}
				\centering
				%\includegraphics[width=1\linewidth]{cat_mesh.png}
			\end{minipage}&
			\begin{minipage}[t]{0.2\linewidth}
				\centering
				%\includegraphics[width=1\linewidth]{cat_square_uniform_mesh.png}
			\end{minipage}&
			\begin{minipage}[t]{0.2\linewidth}
				\centering
				%\includegraphics[width=1\linewidth]{cat_square_uniform.png}
			\end{minipage}\\
			
			%%%%%%%%%%%%%%%%%%%%%%%%%%%%%%%%%%%%%%%%%%%%%
			\begin{minipage}[t]{0.2\linewidth}
				\centering
				%\includegraphics[width=1\linewidth]{david.png}
			\end{minipage}&
			\begin{minipage}[t]{0.2\linewidth}
				\centering
				%\includegraphics[width=1\linewidth]{david_mesh.png}
			\end{minipage}&
			\begin{minipage}[t]{0.2\linewidth}
				\centering
				%\includegraphics[width=1\linewidth]{david_square_uniform_mesh.png}
			\end{minipage}&
			\begin{minipage}[t]{0.2\linewidth}
				\centering
				%\includegraphics[width=1\linewidth]{david_square_uniform.png}
			\end{minipage}\\
			
			%%%%%%%%%%%%%%%%%%%%%%%%%%%%%%%%%%%%%%%%%%%%%
			\begin{minipage}[t]{0.2\linewidth}
				\centering
				%\includegraphics[width=1\linewidth]{face.png}
			\end{minipage}&
			\begin{minipage}[t]{0.2\linewidth}
				\centering
				%\includegraphics[width=1\linewidth]{face_mesh.png}
			\end{minipage}&
			\begin{minipage}[t]{0.2\linewidth}
				\centering
				%\includegraphics[width=1\linewidth]{face_square_uniform_mesh.png}
			\end{minipage}&
			\begin{minipage}[t]{0.2\linewidth}
				\centering
				%\includegraphics[width=1\linewidth]{face_square_uniform.png}
			\end{minipage}\\
			
			
		\end{tabular}
		\caption{Uniform权重,正方形边界的网格参数化与纹理映射}
	\end{table}			
	
	
	\pagebreak
	\subsubsection{Cotangent权重\&单位圆边界}
	\begin{table}[h] % [h] forces the table to be output where it is defined in the code (it suppresses floating)
		\centering % Centre the table
		\begin{tabular}{l l l l}
			\toprule
			\centering
			\textbf{原曲面} & \textbf{原网格} & \textbf{参数化网格} &\textbf{纹理映射}\\
			\midrule
			%%%%%%%%%%%%%%%%%%%%%%%%%%%%%%%%
			\begin{minipage}[t]{0.2\linewidth}
				\centering
				%\includegraphics[width=1\linewidth]{ball.png}
			\end{minipage}&
			\begin{minipage}[t]{0.2\linewidth}
				\centering
				%\includegraphics[width=1\linewidth]{ball_mesh.png}
			\end{minipage}&
			\begin{minipage}[t]{0.2\linewidth}
				\centering
				%\includegraphics[width=1\linewidth]{ball_circle_cotangent_mesh.png}
			\end{minipage}&
			\begin{minipage}[t]{0.2\linewidth}
				\centering
				%\includegraphics[width=1\linewidth]{ball_circle_cotangent.png}
			\end{minipage}\\
			%%%%%%%%%%%%%%%%%%%%%%%%%%%%%%%%%%%%%%%%%%%%%
			\begin{minipage}[t]{0.2\linewidth}
				\centering
				%\includegraphics[width=1\linewidth]{bunny.png}
			\end{minipage}&
			\begin{minipage}[t]{0.2\linewidth}
				\centering
				%\includegraphics[width=1\linewidth]{bunny_mesh.png}
			\end{minipage}&
			\begin{minipage}[t]{0.2\linewidth}
				\centering
				%\includegraphics[width=1\linewidth]{bunny_circle_cotangent_mesh.png}
			\end{minipage}&
			\begin{minipage}[t]{0.2\linewidth}
				\centering
				%\includegraphics[width=1\linewidth]{bunny_circle_cotangent.png}
			\end{minipage}\\
			
			%%%%%%%%%%%%%%%%%%%%%%%%%%%%%%%%%%%%%%%%%%%%%
			\begin{minipage}[t]{0.2\linewidth}
				\centering
				%\includegraphics[width=1\linewidth]{cat.png}
			\end{minipage}&
			\begin{minipage}[t]{0.2\linewidth}
				\centering
				%\includegraphics[width=1\linewidth]{cat_mesh.png}
			\end{minipage}&
			\begin{minipage}[t]{0.2\linewidth}
				\centering
				%\includegraphics[width=1\linewidth]{cat_circle_cotangent_mesh.png}
			\end{minipage}&
			\begin{minipage}[t]{0.2\linewidth}
				\centering
				%\includegraphics[width=1\linewidth]{cat_circle_cotangent.png}
			\end{minipage}\\
			
			%%%%%%%%%%%%%%%%%%%%%%%%%%%%%%%%%%%%%%%%%%%%%
			\begin{minipage}[t]{0.2\linewidth}
				\centering
				%\includegraphics[width=1\linewidth]{david.png}
			\end{minipage}&
			\begin{minipage}[t]{0.2\linewidth}
				\centering
				%\includegraphics[width=1\linewidth]{david_mesh.png}
			\end{minipage}&
			\begin{minipage}[t]{0.2\linewidth}
				\centering
				%\includegraphics[width=1\linewidth]{david_circle_cotangent_mesh.png}
			\end{minipage}&
			\begin{minipage}[t]{0.2\linewidth}
				\centering
				%\includegraphics[width=1\linewidth]{david_circle_cotangent.png}
			\end{minipage}\\
			
			%%%%%%%%%%%%%%%%%%%%%%%%%%%%%%%%%%%%%%%%%%%%%
			\begin{minipage}[t]{0.2\linewidth}
				\centering
				%\includegraphics[width=1\linewidth]{face.png}
			\end{minipage}&
			\begin{minipage}[t]{0.2\linewidth}
				\centering
				%\includegraphics[width=1\linewidth]{face_mesh.png}
			\end{minipage}&
			\begin{minipage}[t]{0.2\linewidth}
				\centering
				%\includegraphics[width=1\linewidth]{face_circle_cotangent_mesh.png}
			\end{minipage}&
			\begin{minipage}[t]{0.2\linewidth}
				\centering
				%\includegraphics[width=1\linewidth]{face_circle_cotangent.png}
			\end{minipage}\\
			
			
		\end{tabular}
		\caption{Cotangent权重,单位圆边界的网格参数化与纹理映射}
	\end{table}			
	
	\pagebreak
	\subsubsection{Cotangent权重\&正方形边界}
	\begin{table}[h] % [h] forces the table to be output where it is defined in the code (it suppresses floating)
		\centering % Centre the table
		\begin{tabular}{l l l l}
			\toprule
			\centering
			\textbf{原曲面} & \textbf{原网格} & \textbf{参数化网格} &\textbf{纹理映射}\\
			\midrule
			%%%%%%%%%%%%%%%%%%%%%%%%%%%%%%%%
			\begin{minipage}[t]{0.2\linewidth}
				\centering
				%\includegraphics[width=1\linewidth]{ball.png}
			\end{minipage}&
			\begin{minipage}[t]{0.2\linewidth}
				\centering
				%\includegraphics[width=1\linewidth]{ball_mesh.png}
			\end{minipage}&
			\begin{minipage}[t]{0.2\linewidth}
				\centering
				%\includegraphics[width=1\linewidth]{ball_square_cotangent_mesh.png}
			\end{minipage}&
			\begin{minipage}[t]{0.2\linewidth}
				\centering
				%\includegraphics[width=1\linewidth]{ball_square_cotangent.png}
			\end{minipage}\\
			%%%%%%%%%%%%%%%%%%%%%%%%%%%%%%%%%%%%%%%%%%%%%
			\begin{minipage}[t]{0.2\linewidth}
				\centering
				%\includegraphics[width=1\linewidth]{bunny.png}
			\end{minipage}&
			\begin{minipage}[t]{0.2\linewidth}
				\centering
				%\includegraphics[width=1\linewidth]{bunny_mesh.png}
			\end{minipage}&
			\begin{minipage}[t]{0.2\linewidth}
				\centering
				%\includegraphics[width=1\linewidth]{bunny_square_cotangent_mesh.png}
			\end{minipage}&
			\begin{minipage}[t]{0.2\linewidth}
				\centering
				%\includegraphics[width=1\linewidth]{bunny_square_cotangent.png}
			\end{minipage}\\
			
			%%%%%%%%%%%%%%%%%%%%%%%%%%%%%%%%%%%%%%%%%%%%%
			\begin{minipage}[t]{0.2\linewidth}
				\centering
				%\includegraphics[width=1\linewidth]{cat.png}
			\end{minipage}&
			\begin{minipage}[t]{0.2\linewidth}
				\centering
				%\includegraphics[width=1\linewidth]{cat_mesh.png}
			\end{minipage}&
			\begin{minipage}[t]{0.2\linewidth}
				\centering
				%\includegraphics[width=1\linewidth]{cat_square_cotangent_mesh.png}
			\end{minipage}&
			\begin{minipage}[t]{0.2\linewidth}
				\centering
				%\includegraphics[width=1\linewidth]{cat_square_cotangent.png}
			\end{minipage}\\
			
			%%%%%%%%%%%%%%%%%%%%%%%%%%%%%%%%%%%%%%%%%%%%%
			\begin{minipage}[t]{0.2\linewidth}
				\centering
				%\includegraphics[width=1\linewidth]{david.png}
			\end{minipage}&
			\begin{minipage}[t]{0.2\linewidth}
				\centering
				%\includegraphics[width=1\linewidth]{david_mesh.png}
			\end{minipage}&
			\begin{minipage}[t]{0.2\linewidth}
				\centering
				%\includegraphics[width=1\linewidth]{david_square_cotangent_mesh.png}
			\end{minipage}&
			\begin{minipage}[t]{0.2\linewidth}
				\centering
				%\includegraphics[width=1\linewidth]{david_square_cotangent.png}
			\end{minipage}\\
			
			%%%%%%%%%%%%%%%%%%%%%%%%%%%%%%%%%%%%%%%%%%%%%
			\begin{minipage}[t]{0.2\linewidth}
				\centering
				%\includegraphics[width=1\linewidth]{face.png}
			\end{minipage}&
			\begin{minipage}[t]{0.2\linewidth}
				\centering
				%\includegraphics[width=1\linewidth]{face_mesh.png}
			\end{minipage}&
			\begin{minipage}[t]{0.2\linewidth}
				\centering
				%\includegraphics[width=1\linewidth]{face_square_cotangent_mesh.png}
			\end{minipage}&
			\begin{minipage}[t]{0.2\linewidth}
				\centering
				%\includegraphics[width=1\linewidth]{face_square_cotangent.png}
			\end{minipage}\\
			
			
		\end{tabular}
		\caption{Cotangent权重,正方形边界的网格参数化与纹理映射}
	\end{table}	
	
	\pagebreak
	\section{总结}
	从实验结果可以看出使用Uniform权重时,曲面表面的纹理会有一定程度的变形,而使用cotangent权重时会改善很多。\\
	曲面参数化的核心是求取重心坐标,除了作业中的Uniform和Cotangent方法之外,还有Watchpress Coordinates、Mean Value Coordinates以及论文\cite{floater1997parametrization}中提到的shape preserving方法,由于时间原因没来得及一一测试。
	
	\bibliographystyle{unsrt}
	\bibliography{bibfile}
	
\end{document}